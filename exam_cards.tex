\documentclass[
  12pt,
  a4paper,
]{article}
% Версия 1.2.0

%%%%%%%%%%%%%%%%%%%%%%%%%%%%%%%%%%%%%%%%%%%%%%%%%%%%%%%%%%%%%%%%%%%%%%%%%%%%%%%
% Настройки по умолчанию
%%%%%%%%%%%%%%%%%%%%%%%%%%%%%%%%%%%%%%%%%%%%%%%%%%%%%%%%%%%%%%%%%%%%%%%%%%%%%%%
\newcommand{\University} % Полное название университета
{
  Федеральное государственное автономное\\
  образовательное учреждение высшего образования\\
  научно-исследовательский институт чародейства и волшебства\\
  (<<НИИЧАВО>>)
}
\newcommand{\Department}{<<Смысла жизни>>} % Кафедра
\newcommand{\HeadOfDepartment}{К.Х.~Хунта} % Заведующий кафедрой
\newcommand{\Course}{Путешествия по выдуманным мирам} % Название дисциплины
\newcommand{\Teacher}{Седловой~Л.~И.} % Преподаватель
\newcommand{\AcademicYear}{2025-2026} % Учебный год
\newcommand{\DecreeNumber}{01.23-45/6ДСП} % Номер приказа
\newcommand{\DecreeDate}{11.11.2025} % Дата приказа

\newcommand{\FileQuestions}{questions.csv} % Файл с вопросами
\newcommand{\NumberOfQuestions}{3} % Количество вопросов в билете: 2 или 3
\newcommand{\OneQuestionMark}{10} % Баллы за первый вопрос/задачу
\newcommand{\TwoQuestionMark}{10} % Баллы за второй вопрос/задачу
\newcommand{\ThreeQuestionMark}{10} % Баллы за третий вопрос/задачу (если есть)

\newcommand{\IsPrintTitlePage}{true} % Печатать ли титульный лист: true или false
\newcommand{\IsPrintCropLine}{true} % Пачатать ли линию разреза: true или false
\newcommand{\RowHeight}{20} % Высота строк таблицы по умолчанию, мм


%%%%%%%%%%%%%%%%%%%%%%%%%%%%%%%%%%%%%%%%%%%%%%%%%%%%%%%%%%%%%%%%%%%%%%%%%%%%%%%
% Используемые пакеты
%%%%%%%%%%%%%%%%%%%%%%%%%%%%%%%%%%%%%%%%%%%%%%%%%%%%%%%%%%%%%%%%%%%%%%%%%%%%%%%
\usepackage[T2A]{fontenc} % Кодировка для международных символов
\usepackage[utf8]{inputenc} % Кодировка UTF-8
\usepackage[english, russian]{babel} % Поддержка русского языка
\usepackage{amsmath,amssymb} % Пакет для математических символов
\usepackage{graphicx} % Пакет для работы с изображениями
\usepackage{tabularray} % Пакет для создания сложных и красивых таблиц
\usepackage{makecell} % Пакет для расширенной работы с таблицами
\usepackage{tikz} % Пакет для рисования линий
\usepackage{array} % Пакет для улучшенной работы с таблицами
\usepackage{datatool} % Пакет для работы с csv
\usepackage{ifthen} % Оператор ветвления
\usepackage{geometry} % Пакет для настройки полей
\geometry{top=1.0cm, bottom=1.0cm, left=1.2cm, right=1.2cm}
\pagenumbering{gobble} % Убрать номера страниц


%%%%%%%%%%%%%%%%%%%%%%%%%%%%%%%%%%%%%%%%%%%%%%%%%%%%%%%%%%%%%%%%%%%%%%%%%%%%%%%
% Загрузка списка вопросов из файла
%%%%%%%%%%%%%%%%%%%%%%%%%%%%%%%%%%%%%%%%%%%%%%%%%%%%%%%%%%%%%%%%%%%%%%%%%%%%%%%
\DTLloaddb[]{questions}{\FileQuestions}


\begin{document}

%%%%%%%%%%%%%%%%%%%%%%%%%%%%%%%%%%%%%%%%%%%%%%%%%%%%%%%%%%%%%%%%%%%%%%%%%%%%%%%
% Титульный лист
%%%%%%%%%%%%%%%%%%%%%%%%%%%%%%%%%%%%%%%%%%%%%%%%%%%%%%%%%%%%%%%%%%%%%%%%%%%%%%%
\title{\textbf{Экзаменационные билеты по дисциплине\\\Course}}
\author{\Teacher}
\date{\AcademicYear}

\ifthenelse{\equal{\IsPrintTitlePage}{true}}
{\maketitle\pagebreak}
{}


%%%%%%%%%%%%%%%%%%%%%%%%%%%%%%%%%%%%%%%%%%%%%%%%%%%%%%%%%%%%%%%%%%%%%%%%%%%%%%%
% Билеты
%%%%%%%%%%%%%%%%%%%%%%%%%%%%%%%%%%%%%%%%%%%%%%%%%%%%%%%%%%%%%%%%%%%%%%%%%%%%%%%
\foreach \num in {1,2,...,\DTLrowcount{questions}}
{

  %----------------------------------------------------------------------------
  % Заголовок билета
  %----------------------------------------------------------------------------
  \noindent
  \begin{center}
    \textbf{ \University }
  \end{center}
  \vspace{-0.5cm}
  \begin{tikzpicture}
    \draw[thick] (0,0) -- (\textwidth,0);
  \end{tikzpicture}
  \begin{center}
    \textbf{ЭКЗАМЕНАЦИОННЫЙ БИЛЕТ № \num}\\
    \textbf{По дисциплине <<\Course>>}
  \end{center}

  %----------------------------------------------------------------------------
  % Таблица с вопросами
  %----------------------------------------------------------------------------

  % TODO: Надо устранить дублирование в таблице, но не получается вставить
  % условие внутрь таблицы
  \ifthenelse{\equal{\NumberOfQuestions}{2}}
  { % Если 2 вопроса в билете
    \begin{table}[h!]
      \centering
      \begin{tblr}
      {
        width = \linewidth,
        colspec = {Q[37]Q[693]Q[210]},
        column{1} = {c},
        column{3} = {c},
        row{1} = {c, m, font=\bfseries},
        row{2-4}={\RowHeight mm},
        hlines,
        vlines,
      }
        № & Вопрос / задача & Максимальная оценка в баллах \\
        1 & \DTLgetvalue{\Question}{questions}{\num}{1}\Question & \OneQuestionMark \\
        2 & \DTLgetvalue{\Question}{questions}{\num}{2}\Question & \TwoQuestionMark
      \end{tblr}
    \end{table}
  }
  { % Если 3 вопроса в билете
    \begin{table}[h!]
      \centering
      \begin{tblr}
      {
        width = \linewidth,
        colspec = {Q[37]Q[693]Q[210]},
        column{1} = {c},
        column{3} = {c},
        row{1} = {c, m, font=\bfseries},
        row{2-4}={\RowHeight mm},
        hlines,
        vlines,
      }
        № & Вопрос / задача & Максимальная оценка в баллах \\
        1 & \DTLgetvalue{\Question}{questions}{\num}{1}\Question & \OneQuestionMark \\
        2 & \DTLgetvalue{\Question}{questions}{\num}{2}\Question & \TwoQuestionMark \\
        3 & \DTLgetvalue{\Question}{questions}{\num}{3}\Question & \ThreeQuestionMark
      \end{tblr}
    \end{table}
  }

  %----------------------------------------------------------------------------
  % Подвал
  %----------------------------------------------------------------------------
  \vspace{-0.5cm} % Удаляем разрыв между таблицей и текстом
  \noindent
  \begin{tikzpicture}
    \draw[thick] (0,0) -- (\textwidth,0);
  \end{tikzpicture}
  \noindent
  \footnotesize
  \textit
  {
    Билет рассмотрен и утвержден на заседании кафедры
    от \DecreeDateг. протокол № \DecreeNumber
  }

  \vspace{0.9cm}
  \noindent
  \textbf{Заведующий кафедрой \Department \hfill \HeadOfDepartment}

  %----------------------------------------------------------------------------
  % Линия отреза
  %----------------------------------------------------------------------------
  \ifthenelse{\equal{\IsPrintCropLine}{true}}
  {
    \noindent
    \begin{center}
      \begin{tikzpicture}
        \draw[dashed] (0,0) -- (\textwidth,0);
      \end{tikzpicture}
    \end{center}
  }
  {}
  \pagebreak

}
\end{document}
